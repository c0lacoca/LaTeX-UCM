El grafeno, un material bidimensional basado en el carbono, es uno de los materiales más investigados de las últimas décadas. Las extensas propiedades del grafeno lo convierten en el objeto de estudio de multitud de disciplinas de la física, ya sea por su extraordinaria resistencia mecánica o por sus propiedades electrónicas y ópticas, entre otras. En este trabajo de fin de grado, estudiamos de cerca las propiedades mecánicas del grafeno, como su elasticidad, y sus propiedades eléctricas y cómo cambian cuando se estira una red de grafeno, además de observar las diferentes estructuras que puedan formarse tras producirse una ruptura. El método de estiramiento influye en gran medida en la estructura final y sus propiedades. En este trabajo, consideramos estiramientos en la dirección \emph{armchair} con diferentes condiciones de temperatura y de contorno impuestas sobre el grafeno, así como la presencia de impurezas (boro y nitrógeno) y de defectos (vacantes y divacantes) en la red. Este análisis se realiza de forma computacional con un código (FIREBALL) basado en la Teoría del Funcional de la Densidad (DFT).
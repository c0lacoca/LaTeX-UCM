Graphene, a two-dimensional carbon-based material, is one of the most researched materials in recent decades. The extensive properties of graphene make it the subject of study across many disciplines, whether it is due to its extraordinary mechanical strength and flexibility or its electronic and optical properties, among others. In this Bachelor's Degree final project, we closely examine the mechanical properties of graphene, such as its elasticity and toughness, as well as its electrical properties, like reactivity, and how they change when a graphene lattice is strained through stretching, in addition to observing different structures that may form after a rupture occurs. The method of stretching greatly influences the material's properties. In this work, we consider stretching in the \emph{armchair} direction with different temperature and boundary conditions imposed on the lattice, as well as the presence of impurities (boron and nitrogen) and defects (vacancies and divacancies). This analysis is performed computationally using a DFT-based (Density Functional Theory) code (FIREBALL).\\


% Graphene, a two-dimensional carbon-based material, is one of the most researched materials. The extensive properties of graphene make it the object of study of  of physics, either for its extraordinary mechanical strength or for its electronic and optical properties, among others. In this thesis, we closely study the mechanical properties of graphene, such as its elasticity and toughness, and its electrical properties and how they change when a graphene network is stretched, as well as the different structures that can be formed after a rupture occurs. The method of stretching greatly influences the properties of graphene. In this work, we consider stretching in the \emph{armchair} direction with different temperature and boundary conditions imposed on the graphene, as well as the presence of impurities (boron and nnitrogen) and defects (vacancies and divacancies) in the lattice. This analysis is performed computationally with a code (FIREBALL) based on Density Functional Theory (DFT).
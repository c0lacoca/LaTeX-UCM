\documentclass{article}

\usepackage[T1]{fontenc}


\renewcommand\familydefault{\rmdefault}  %% fuente 
\usepackage[a4paper, left=2.5cm, right= 2.5cm, top=2.5cm, bottom= 2.5cm]{geometry} %cambiar los márgenes
\usepackage{ amssymb, cancel, extarrows, amsmath, amsthm }%símbolos matemáticos
\usepackage{wasysym}  %la carita triste
\usepackage{derivative}
\usepackage{physics}

\newcommand{\D}[2][1]{\odif[order={#1}]{#2}} %% Notación de diferenciales con paquete derivative
\newcommand{\B}[1]{\mathbf{#1}}  %% Para hacer negrita sin escribir mathbf
\newcommand{\K}[2][]{\mathbb{#2}^{#1}} %%Letras chulas de espacios vectoriales y cuerpos
\newcommand{\Cl}[1]{\mathcal{#1}}  %%Letras caligráficas
\newcommand{\cte}{\text{const.}}

\usepackage{graphicx, wrapfig}% Include figure files
\usepackage{subfig}
\usepackage{dcolumn}% Align table columns on decimal point
\usepackage{bm}% bold math
\usepackage{hyperref}% add hypertext capabilities
\usepackage[spanish, es-tabla]{babel} %traducción al español
\addto{\captionspanish}{%
  \renewcommand*{\chaptername}{CAPÍTULO}
  \renewcommand*{\pagename}{PÁGINA}}
\usepackage{units}
\usepackage{fancyhdr, ragged2e}
\usepackage{anyfontsize}
\setlength{\headheight}{13.07002pt}
\renewcommand{\footrulewidth}{0.4pt}
\usepackage{float}
\usepackage{booktabs}
\usepackage[11pt]{moresize}
\usepackage{lipsum}
\usepackage{changepage}
\usepackage{calc}

\usepackage{multirow}
\usepackage{multicol}
\usepackage[table,xcdraw]{xcolor}
\usepackage{xcolor, colortbl}
\usepackage{pdfpages} %%insertar pdf dentro del documento
\usepackage{pagecolor}
\usepackage{tcolorbox}
\newtcolorbox{mybox}{colback=gray!5!white,colframe=black!75!black}
\definecolor{cadetblue4}{rgb}{.33,.53,.55}
\definecolor{azulcrepuscular}{rgb}{.49,.62,.75}
\definecolor {atomictangerine} {rgb} {1.0, 0.6, 0.4}
\definecolor{portada}{RGB}{2, 2, 0}
\usepackage{longtable}
\usepackage{hyperref}
\hypersetup{
    colorlinks=true,
    linkcolor=black,
    filecolor=magenta,      
    urlcolor=blue,
    pdftitle={Mec. Teo: Entregable 3},
    pdfpagemode=FullScreen,
    }

\usepackage{titlesec}
\titleformat{\chapter}[display]
  {\normalfont\rmfamily\huge\bfseries}
  {\chaptertitlename\ \thechapter}{20pt}{\Huge}[{\titlerule[2pt]}]
\titleformat*{\section}{\Large\normalfont\rmfamily\bfseries}
\titleformat*{\subsection}{\large\normalfont\rmfamily\bfseries}
\titleformat*{\subsubsection}{\normalsize\normalfont\rmfamily\bfseries} 

\usepackage{notomath}   %% fuente para entornos matemáticos

\setlength{\parindent}{0cm}  %% para quitar la sangría francesa esa repulsiva

\newcounter{define}[section] %%contador para definiciones

\newenvironment{define}[1][]{\refstepcounter{define}\underline{\textbf{Definición \thedefine}} [#1]: \ \begin{minipage}[t]{\textwidth-\widthof{\underline{\textbf{Definición \thedefine}}["\emph{#1}"]:}}}{\end{minipage}\\\\}  %%definicion de entorno de definicion

\newcounter{postula}[section] %%contador para postulados

\newenvironment{postula}[1][]{\begin{mybox} \refstepcounter{postula}\underline{\textbf{Postulado \thepostula}} [#1]: \ \begin{minipage}[t]{\textwidth-\widthof{\underline{\textbf{Postulado \thepostula}}["\emph{#1}"]:}}}{\end{minipage}\end{mybox}}  %%definicion de entorno de postulado

% Algún paquete más de Sergio :P
% \usepackage{amsmath}
%% ==>te lo he movido a la línea 6 ;) -Manu

\begin{document}

\begin{center}
   \HUGE\textbf{ \scshape Física Atómica y Molecular}  \\ \huge Entregable de la parte de moléculas \\ \vspace{0.5cm}\today \\
   \vspace{0.7cm}
   \large Manuel Lozano Bermúdez, Sergio Zucchi Mesia.\\ \rule{\textwidth}{2pt} 
\end{center}

\section*{Ejercicio 1}
\begin{minipage}{0.72\textwidth}
La tabla adjunta muestra los orígenes de las 3 primeras bandas de la progresión 0-v del sistema $B^2\Pi_g - X^2\Sigma_u{}^+$ del Be$_2{}^+$ con el isótopo estable $^9$Be (A=9,Z=4). Se conoce además que la constante rotacional para el estado electrónico fundamental de esta molécula es $B_r = 0.765$ cm$^{-1}$.
\end{minipage}
\begin{minipage}{.3\textwidth}
\centering
        \begin{tabular}{|c|c|} \hline 
            Banda & $E_\text{origen} $ (cm$^{-1}$) \\ \hline 
             0-0&$29350.3$  \\ \hline 
             0-1&$28833.3$  \\ \hline 
             0-2&$28325.2$  \\  \hline 
        \end{tabular}
        \label{tab:ej1}
\end{minipage}


\begin{enumerate}
    \item[a)] Discutir la estabilidad de las especies Be$_2$ y Be$_2{}^+$ en sus configuraciones electrónicas fundamentales en términos del orden de enlace. Supóngase el orden típico de energías de los orbitales moleculares en moléculas diatómicas homonucleares. 

    \item[b)] Deducir cuál es el primer estado electrónico excitado del Be$_2{}^+$ (no es el estado $B^2 \Pi_g$) indicando los números cuánticos $\Lambda$ y $S$ así como sus simetrías. ¿Es posible una transición E1 entre este primer estado electrónico excitado y el estado electrónico fundamental?

    \item[c)] Obtener la constante armónica y la corrección anarmónica en cm$^{-1}$ para el estado electrónico fundamental del Be$_2{}^+$.

    \item[d)] Obtener la distancia internuclear de equilibrio en ángstrom para el estado electrónico fundamental del Be$_2{}^+$. ¿Qué valores tendrían la distancia de equilibrio y la constante rotacional si uno de los dos núcleos fuera $^{10}$Be? 
\end{enumerate}

\section*{Ejercicio 2}
Se conoce que la molécula Be$_3$ es estable y, en su estado fundamental, los 3 átomos de Be forman un triángulo equilátero. Deducir el número de constantes rotacionales necesarias para describir los niveles rotacionales de la molécula y si la molécula presenta espectro rotacional puro en los dos siguientes supuestos:
\begin{enumerate}
    \item[a)] Los tres núcleos son $^9$Be.

    \item[b)] Dos núcleos son $^9$Be mientras que el tercero es $^{10}$Be, manteniéndose la geometría de la molécula.
\end{enumerate}
.\dotfill 

%% \textbf{Va Manu, nosotros podemos \smiley{}}
\paragraph{Ej. 1} 
\begin{enumerate}
    \item[a)] 
      %% lo pongo en rojo porq todavía no está corregido
    Orden de enlace: $OE= (n^\text{ligantes} - n^\text{antiligantes})/2$. Necesitamos conocer el nº de electrones en estados ligantes y en antiligantes. Una vez los tengamos, cuanto mayor sea el orden de enlace, más estable será la molécula. El balance neto impide la formación. \\
    
    El orden de llenado para moléculas es el siguiente: 1s$\sigma_g$ 1s$\sigma_u{}^*$ 2s$\sigma_g$ 2s$\sigma_u{}^*$ 2p$\pi_u$ 2p$\sigma_g \ldots$
    \begin{flalign*}
         \text{Be}_2{}^+ &\to \uparrow \downarrow \quad \uparrow \downarrow\quad \uparrow \downarrow \quad \uparrow \ \  \implies OE = 1/2&&\\
        \text{Be}_2\phantom{{}^+}     &\to \uparrow \downarrow \quad \uparrow \downarrow\quad \uparrow \downarrow \quad \uparrow \downarrow \implies OE = 0&&
    \end{flalign*}
    Observamos que el Be$_2$ forma parejas completas de electrones ligantes y antiligantes, de manera que {\bfseries no puede formar estados ligados}. Por otro lado, el Be$_2{}^+$ tiene un solo electrón en el estado antiligante en 2s$\sigma_u{}^*$, de manera que {\bfseries sí se forma la molécula}.
    

    \item[b)] 
    
    El estado excitado podría ser el electrón en el siguiente estado de menor energía. En ese caso, sería el estado 2p$\pi_u$, con el número cuántico $\Lambda = 1$ y $S = 1/2$. Por ello, el estado, denotado como $^{2S+1}\Lambda$, es $^2\Pi$. Además, como nos dicen que el estado excitado no tiene paridad gerade, el estado final será $^2\Pi_u$.\\

    Las transiciones E1 de estados electrónicos imponen una regla de selección sobre la paridad haciendo que ésta deba cambiar. Tanto el estado excitado como el fundamental $^2\Sigma_u{}^+$ tienen simetría ungerade, de manera que esta transición E1 \textbf{no es posible}. 
    

    \item[c)]
    
    Los niveles de energía para una molécula son:
    $$
    E_{n,\nu ,N} = \Cl{V}_n(R_0) + \underbrace{\hbar\omega_0(\nu + 1/2)}_{\text{aprox. armónica}} + \underbrace{B_rN(N+1)}_{\text{{rotor rígido}}} - \underbrace{\hbar\omega_0 \beta (\nu + 1/2)^2}_{\text{anarmonicidad}}- a(\nu + 1/2)N(N+1) - bN^2(N+1)^2
    $$
    Considerando únicamente las energías vibracionales:
    $$
    E_{n,\nu} = E_n{}^\text{el} + \hbar \omega_0(\nu + 1/2) - \hbar \omega_0 \beta (\nu + 1/2)
    $$
    

    Usando los datos de la tabla
    \begin{equation*}
        \begin{cases}
            E[0-0]-E[0-1]=517.0\,\text{cm}^{-1}=\hbar\omega_0-2\left[-\hbar\omega_0\beta\right]\\
            E[0-0]-E[0-2]=1025.1\,\text{cm}^{-1}=\hbar\omega_0-6\left[-\hbar\omega_0\beta\right]\\
            E[0-1]-E[0-2]=508.1\,\text{cm}^{-1}=\hbar\omega_0-4\left[-\hbar\omega_0\beta\right]
        \end{cases}    
    \end{equation*}
    De esta forma la constante anarmónica ($-\hbar\omega_0\beta$) y la constante armónica ($\hbar\omega_0$) son:
    \[
    \boxed{-\hbar\omega_0\beta=8.9\,\text{cm}^{-1}}\qquad;\qquad\boxed{\hbar\omega_0=534.8\,\text{cm}^{-1}}
    \]

    \item[d)] {
    Conociendo que la constante rotacional en el estado fundamental es $B_r=0.765$ cm$^{-1}$ y que esta viene dada por 
    \[B_r=\frac{\hbar^2}{2\mu R_0^2}\Rightarrow R_0=\frac{\hbar^2}{\sqrt{2B_r\mu}}\]
    donde $\mu=(N_n m_n) (N_p m_p)/\left[(N_p m_p)+(N_n m_n)\right]$ es la masa reducida del núcleo y $R_0$ es la distancia internuclear de equilibrio.

    Por simplicidad de los cálculos, se han usado unidades atómicas con $m_e=\hbar=e=1$ y $m_n=m_p=m=1822\,m_e$. 
    %% fun fact: masa e-/masa p+ es aprox 6·pi^5.
    
    Si consideramos que los dos núcleos de $^{9}\text{Be}$ entonces $\mu=\frac{9}{2}m\approx8199$, usando la fórmula de la distancia internuclear
    \[
    \boxed{R_0\left[^{9}\text{Be}\right]=\frac{1}{\sqrt{2B_r\mu}}=0.89284\,\text{\AA}}
    \]

    Por otro lado, si uno de los núcleos fuese $^{10}\text{Be}$ entonces $\mu=\frac{90}{19}m\approx8631$. Cambiando la masa reducida, la constante rotacional cambia como 
    \[
    \frac{B_r\left[^{9}\text{Be}\right]}{B_r\left[^{10}\text{Be}\right]}=\frac{\mu\left[^{10}\text{Be}\right]}{\mu\left[^{9}\text{Be}\right]}\Rightarrow \boxed{B_r\left[^{10}\text{Be}\right]=0.72671\,\text{cm}^{-1}}
    \]
    Usando esta constante rotacional, el radio internuclear de equilibrio será
    \[
    \boxed{R_0\left[^{9}\text{Be}\right]=0.89287\,\text{\AA}}
    \]
    Cabe notar que si uno de los núcleo tuviese un neutrón más, la distancia de equilibrio y la constante rotacional no cambian de forma considerable.
    }
\end{enumerate}

\paragraph{Ej. 2}

Teniendo una molécula con 3 átomos con una geometría de triángulo equilátero entonces el número de grados de libertad (modos normales o constantes rotacionales) que se necesitan para describirla son
\[
    \underbrace{3}_{\text{nº átomos}}\cdot\underbrace{3}_{\text{g.l. núcleo}}-\underbrace{3}_{\text{traslación}}-\underbrace{3}_{\text{rotacional}}=3\,\text{grados de libertad}
\]
\begin{enumerate}
    \item[a)] {
    Teniendo 3 átomos de Be iguales, se puede considerar que la molécula es un \textbf{tropo esférico} por lo que se puede describir la molécula usando solo \textbf{1 constante rotacional}. Al tener 3 átomo iguales, la molécula no va a tener momento dipolar y por eso no tiene espectro rotacional puro.
    }

    \item[b)] {
    Siendo uno de los átomos diferentes, se puede considerar que la molécula es un \textbf{trompo simétrico}, se describe con \textbf{2 constantes rotacionales} y puede tener un espectro rotacional puro.
    }
\end{enumerate}
\end{document}

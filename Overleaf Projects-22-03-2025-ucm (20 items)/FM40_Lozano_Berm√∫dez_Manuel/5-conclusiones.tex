\newpage
\section{Conclusiones}

A lo largo de este trabajo, hemos revisado los fundamentos físicos de la Teoría del Funcional de la Densidad, además de las aproximaciones principales realizadas, y hemos puesto en práctica el modelo estudiando el efecto del estiramiento de una red de grafeno en sus propiedades mecánicas y electrónicas. Utilizando el código FIREBALL, basado en la teoría DFT y la dinámica molecular, y un modelo sencillo de desplazamiento de la red atómica, hemos reproducido el efecto del estiramiento de la red y hemos deducido diversas propiedades mecánicas de este material, como el módulo de Young, y comparado con éxito con resultados experimentales y de otras simulaciones realizadas con otro programa de cálculo utilizado dentro del marco teórico de la DFT, Quantum Espresso, que, si bien es más preciso, requiere muchos más recursos para realizar el cálculo (típicamente realizado en supercomputadores), a diferencia del programa FIREBALL, que puede realizar los cálculos desde un ordenador de sobremesa. Hemos estudiado diferentes técnicas de estiramiento y su estabilidad, además de analizar el efecto de los dopantes (boro y nitrógeno) y los defectos estructurales (vacante y divacante) e interpretando las diferentes estructuras resultantes, como cadenas de átomos de carbono. A su vez, hemos calculado la densidad de estados en diferentes etapas de los estiramientos, apreciando un cambio importante respecto a la densidad de estados del grafeno ideal alrededor del nivel de Fermi en donde parece formarse un gap, como ha sido posible confirmar en experimentos. Por último, también hemos propuesto reactividad en las cadenas monoatómicas al analizar la densidad de estados. \\


% En experimentos ha sido posible confirmar una apertura de gap, por lo que sería necesario mejorar el cálculo de la densidad de estados para confirmar este resultado con nuestro modelo de estiramiento
% Cabe comentar los resultados del módulo de Young obtenidos. Si bien los ajustes realizados brindan valores del módulo de Young coherentes con los experimentales y son compatibles con los obtenidos mediante otros modelos de simulación, la poca cantidad de datos utilizados junto con la pobreza del ajuste ($R^2 = 0.6$) no nos permiten afirmar con confianza que sean resultados fiables. Si bien es cierto que, para pequeñas fuerzas aplicadas y pequeñas deformaciones, los materiales presentan un régimen lineal, un análisis más exhaustivo debería tener en cuenta la respuesta mecánica no lineal del grafeno. Este modelo, adoptado por ejemplo en el resultado de Lee \cite{science}, tiene en cuenta órdenes superiores de deformación, teniendo que la tensión mecánica en la dirección \arm es
% \begin{equation}
%     \sigma = Y \varepsilon + D \varepsilon^2
% \end{equation}
% donde $Y$ es el módulo de Young y $D$ el módulo elástico de tercer orden. Esta implementación de modelo de elasticidad sería interesante para realizar trabajos futuros, pero por cuestiones de tiempo y de dificultad no ha sido posible introducirlo en este trabajo. 

En general, para futuros trabajos a partir de este proyecto se abre un gran abanico de posibilidades, como determinar en qué condiciones específicas se forman las cadenas monoatómicas al romper mediante dinámica molecular, así como estudiar la capacidad de captación de átomos o de moléculas por estas cadenas monoatómicas. Como conclusión final a esta memoria, este trabajo es la convergencia de muchas líneas de trabajo, de ideas propuestas y, sobre todo, de muchos errores e intentos fallidos; que han sido las lecciones de las cuales más he aprendido en el desarrollo de este proyecto. Más allá de la física aprendida o de los resultados obtenidos, me ha proporcionado una exposición de primera mano a la labores de investigación básicas que se realizan en todos los campos de la física y, en general, de la ciencia. Por último, representa tanto un punto final a mis estudios de grado como un nuevo punto de partida para seguir formándome en el área de la física computacional. 

\newpage
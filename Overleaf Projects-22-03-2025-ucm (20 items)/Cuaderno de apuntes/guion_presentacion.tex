\chapter{Presentación}
\newcommand{\pasa}{{\color{red} Pasa pagina} \\}
\large
Buenos días, soy Manuel Lozano, y os voy a hablar un poco sobre mi trabajo de fin de grado, que se titula \emph{Estirando el grafeno}. \\

\pasa

En primer lugar, os hablaré un poco sobre la motivación y los objetivos principales del proyecto y luego pasaré a hablaros del modelo teórico y las herramientas utilizadas para el desarrollo del trabajo. Finalmente, introduciré los cálculos iniciales y los resultados obtenidos a lo largo de estos meses, terminando con las conclusiones principales sacadas. \\

\pasa

\section{Motivación y objetivos iniciales}

El grafeno, como muchos de vosotros ya sabréis, ha sido y es uno de los materiales más investigados de las últimas décadas. Fue sintetizado por primera vez en 2004 por los físicos Andre Geim y Konstantin Novoselov y, desde entonces, se han ido explorando sus múltiples aplicaciones, tanto mecánicas y electrónicas como muchas otras. \\

Ya en 2008 se han estudiado estas propiedades del grafeno, como las constantes elásticas y en concreto el módulo de Young, fijándolo en $1.0 \pm 0.1 $ TPa, confirmando que se trata de un material muy resistente al estiramiento. Otra de las propiedades más destacables es la de los famosos electrones relativistas sin masa de Dirac, debidos a una cancelación exacta de la densidad de estados en una región de energías próximas al nivel de Fermi, que se conoce como conos de Dirac, como se puede ver en la figura. Estos conos de dirac hacen que el grafeno tenga gap de semiconductor nulo, lo cual no es muy bueno si quieres hacer electrónica, pero se ha podido comprobar teórica (y experimentalmente) que estirando en cualquiera de las direcciones del grafeno puede abrirse este gap de semiconductor. \\

\pasa

\section{Objetivos}

Por tanto los objetivos de este proyecto son entender estas propiedades mecánicas del grafeno e intentar estimar sus constantes de elasticidad, así como determinar las estructuras más estables tras los estiramientos utilizando diferentes técnicas para estirar en la dirección armchair de la red hexagonal, como se puede ver en la imagen. Se simulan estos estiramientos a temperatura de cero absoluto y también a 600 K para simular un estiramiento más ``realista''. También dopamos la red y creamos defectos como huecos o vacantes para analizar el efecto que tienen en la estabilidad de las estructuras. Finalmente, también calcularemos la densidad de estados en diferentes regiones de la red y en diferentes etapas del estiramiento para determinar la apertura de gap y otros efectos interesantes.\\

\pasa 

\section{Fundamento teórico}

Primero comentar los aspectos principales del método FIREBALL y la teoría en la que está basada: la DFT o la Teoría del Funcional de la Densidad. La teoría DFT es un modelo físico usado ampliamente en la física-química e ingeniería para realizar cálculos de la estructura atómica y electrónica de un conjunto de átomos o moléculas. Esta teoría, entre otras aproximaciones, se basa en dos teoremas matemáticos conocidos como teoremas de Hohenberg-Kohn para resolver la ecuación de Schrödinger del sistema completo en un formalismo matemático basado en la densidad electrónica. Con este formalismo, puede simplificarse el problema pasando a resolver las ecuaciones de Kohn-Sham para electrones individuales. \\

De esta teoría se alimenta nuestro programa de cálculo, llamado FIREBALL, que se basa en un formalismo de tight-binding (o de electrones fuertemente enlazados) con pseudo-potenciales. En esta aproximación de pseudo-potencial incluimos los electrones de capas más internas con el núcleo atómico para formar un core. Esto permite utilizar combinaciones lineales de orbitales atómicos con radios de corte, para los cuales estos orbitales se anulan por debajo de ese valor. También utilizamos otra aproximación de cálculo de interacciones entre electrones a tres cuerpos, lo cual nos permite reducir significativamente el tiempo de cálculo de las simulaciones. \\

\pasa

\section{Grafeno 1x1}

Centrándonos ya en el grafeno, es conveniente comprobar que el programa de simulación utiliza valores correctos de la distancia entre átomos de carbono de la basa de grafeno, que se ve en esta imagen. Para ello, realizamos simulaciones de dos átomos de carbono a diferentes distancias interatómicas para determinar cuál es la de mínima energía. Esto nos devuelve un valor de 1.43 \AA \ muy razonable comparado con el valor experimental de 1.42 \AA . \\

\pasa 

Podemos terminar de confirmar que se reproducen correctamente las propiedades electrónicas del grafeno si nos fijamos en la densidad de estados en cualquiera de los átomos de la base, y confirmamos que tenemos el famoso cono de Dirac del que hablábamos antes. 

\section{Resultados}

Finalmente ya estamos en disposición de simular una red de grafeno en condiciones. En nuestro caso, simularemos una red de 288 átomos de carbono de tamaño 18x8. Para estirar esta red, consideraremos 4 formas de estiramiento: 
\begin{enumerate}
    \item Por los extremos de forma rígida. 
    \item Por el centro de forma rígida.
    \item Por los extremos de forma rígida y progresivamente desde el centro. 
    \item Por los extremos y añadiendo pasos de dinámica molecular.
\end{enumerate}
Repito que estiraremos únicamente en la dirección armchair y que usaremos dopantes y defectos a la vez que estiramos. \\

\pasa




\usepackage[T1]{fontenc}
\renewcommand\familydefault{\sfdefault}

%%%%%%%% MATES %%%%%%%%%%%%%%%%%%%%%5

\usepackage[a4paper, left=2.5cm, right= 2.5cm, top=2.5cm, bottom= 2.5cm]{geometry} %cambiar los márgenes
\usepackage{ amssymb, cancel, extarrows, amsmath, amsthm }%símbolos matemáticos
\usepackage{wasysym}  %la carita feliz y triste
\usepackage{derivative}
\usepackage{physics}
\usepackage{bm}% bold math
\usepackage{units}

\newcommand{\D}[2][1]{\odif[order={#1}]{#2}} %% Notación de diferenciales con paquete derivative
\newcommand{\B}[1]{\mathbf{#1}}  %% Para hacer negrita sin escribir mathbf
\newcommand{\K}[2][]{\mathbb{#2}^{#1}} %%Letras chulas de espacios vectoriales y cuerpos
\newcommand{\Cl}[1]{\mathcal{#1}}  %%Letras caligráficas

%%%%%%%%%%%%% FLOATS, BABEL Y ESTILO DE PAGINA %%%%%%%%%%%%%%%%%%%%%%%%%

\usepackage{graphicx, wrapfig}% Include figure files
\usepackage{subfig}
\usepackage{dcolumn}% Align table columns on decimal point
\usepackage[spanish, es-tabla]{babel} %traducción al español
\addto{\captionspanish}{%
  \renewcommand*{\chaptername}{CAPÍTULO}
  \renewcommand*{\pagename}{PÁGINA}}
\usepackage{fancyhdr, ragged2e}
\usepackage{anyfontsize}
\setlength{\headheight}{13.07002pt}
\renewcommand{\footrulewidth}{0.4pt}
\usepackage{float}
\usepackage{booktabs}

\usepackage{multirow}
\usepackage{multicol}
\usepackage[table,xcdraw]{xcolor}
\usepackage{xcolor, colortbl}
\usepackage{pdfpages} %%insertar pdf dentro del documento
\usepackage{pagecolor}
\usepackage{coffeestains}
\usepackage{tcolorbox}
\newtcolorbox{mybox}{colback=gray!5!white,colframe=black!75!black}
\definecolor{cadetblue4}{rgb}{.33,.53,.55}
\definecolor{azulcrepuscular}{rgb}{.49,.62,.75}
\definecolor {atomictangerine} {rgb} {1.0, 0.6, 0.4}
\usepackage{longtable}
\usepackage{hyperref}
\hypersetup{
    colorlinks=true,
    linkcolor=black,
    filecolor=magenta,      
    urlcolor=blue,
    pdftitle={Overleaf Example},
    pdfpagemode=FullScreen,
    }

\usepackage{titlesec}   %%Títulos customizados
\titleformat{\chapter}[display]
  {\normalfont\rmfamily\huge\bfseries}
  {\chaptertitlename\ \thechapter}{20pt}{\Huge}[{\titlerule[2pt]}]  %%Títulos a familia \rmfamily . Generar linea gruesa en los capítulos
\titleformat*{\section}{\Large\normalfont\rmfamily\bfseries}
\titleformat*{\subsection}{\large\normalfont\rmfamily\bfseries}
\titleformat*{\subsubsection}{\normalsize\normalfont\rmfamily\bfseries} 


\usepackage{notomath} %lo pongo aquí porque si no entra en conflicto con otros paquetes. Para usar \sfdefault

\setlength{\parindent}{0cm}  %% para quitar la sangría francesa esa repulsiva
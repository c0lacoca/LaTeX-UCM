\documentclass{article}

\usepackage[T1]{fontenc}
\renewcommand\familydefault{\sfdefault}
\usepackage[a4paper, left=2.5cm, right= 2.5cm, top=2.5cm, bottom= 2.5cm]{geometry} %cambiar los márgenes
\usepackage{ amssymb, cancel, extarrows, amsmath, amsthm }%símbolos matemáticos
\usepackage{wasysym}  %la carita triste
\usepackage{derivative}
\usepackage{physics}

\newcommand{\D}[2][1]{\odif[order={#1}]{#2}} %% Notación de diferenciales con paquete derivative
\newcommand{\B}[1]{\mathbf{#1}}  %% Para hacer negrita sin escribir mathbf
\newcommand{\K}[1]{\mathbb{#1}} %%Letras chulas de espacios vectoriales y cuerpos
\newcommand{\Cl}[1]{\mathcal{#1}}  %%Letras caligráficas
\DeclareMathOperator{\diag}{diag}

\usepackage{graphicx, wrapfig}% Include figure files
\usepackage{subfig}
\usepackage{dcolumn}% Align table columns on decimal point
\usepackage{bm}% bold math
\usepackage{hyperref}% add hypertext capabilities
\usepackage[spanish, es-tabla]{babel} %traducción al español
\addto{\captionspanish}{%
  \renewcommand*{\chaptername}{CAPÍTULO}
  \renewcommand*{\pagename}{PÁGINA}}
\usepackage{units}
\usepackage{fancyhdr, ragged2e}
\usepackage{anyfontsize}
\setlength{\headheight}{13.07002pt}
\renewcommand{\footrulewidth}{0.4pt}
\usepackage{float}
\usepackage{booktabs}
\usepackage[11pt]{moresize}
\usepackage{lipsum}
\usepackage{changepage}
\usepackage{calc}

\usepackage{multirow}
\usepackage{multicol}
\usepackage[table,xcdraw]{xcolor}
\usepackage{xcolor, colortbl}
\usepackage{pdfpages} %%insertar pdf dentro del documento
\usepackage{pagecolor}
\usepackage{tcolorbox}
\newtcolorbox{mybox}{colback=gray!5!white,colframe=black!75!black}
\definecolor{cadetblue4}{rgb}{.33,.53,.55}
\definecolor{azulcrepuscular}{rgb}{.49,.62,.75}
\definecolor {atomictangerine} {rgb} {1.0, 0.6, 0.4}
\definecolor{portada}{RGB}{2, 2, 0}
\usepackage{longtable}
\usepackage{hyperref}
\hypersetup{
    colorlinks=true,
    linkcolor=black,
    filecolor=magenta,      
    urlcolor=blue,
    pdftitle={Overleaf Example},
    pdfpagemode=FullScreen,
    }

\usepackage{titlesec}

\titleclass{\part}{top}
\titleformat{\part}[display]{\titlerule[2pt]\centering\scshape\rmfamily\HUGE}{\bfseries Parte \thepart }{0pt}{}[{\titlerule[2pt]}] 

\titleformat{\chapter}[display]
  {\normalfont\rmfamily\huge\bfseries}
  {\chaptertitlename\ \thechapter}{20pt}{\Huge}[{\titlerule[2pt]}]
\titleformat*{\section}{\Large\normalfont\rmfamily\bfseries}
\titleformat*{\subsection}{\large\normalfont\rmfamily\bfseries}
\titleformat*{\subsubsection}{\normalsize\normalfont\rmfamily\bfseries} 


\usepackage{notomath}

\setlength{\parindent}{0cm}  %% para quitar la sangría francesa esa repulsiva


\newcounter{define}[section] %%contador para definiciones
\renewcommand\thedefine{%
\thechapter.%
\ifnum\value{section}>0 \arabic{section}.\fi
\arabic{define}}

\newenvironment{define}{\refstepcounter{define}\underline{\textbf{Definición \thedefine}}: \ \begin{minipage}[t]{\textwidth-\widthof{\underline{\textbf{Definición \thedefine}}:\ }}}{\end{minipage}\\\\}  %%definicion de entorno de definicion

\newcounter{postula}[subsection] %%contador para postulados
\renewcommand\thepostula{%
\thechapter.%
\ifnum\value{subsection}>0 \arabic{subsection}.\fi
\arabic{postula}}

\newenvironment{postula}[1][]{\begin{mybox} \refstepcounter{postula}\textbf{\underline{Postulado \thepostula}.} \ \begin{minipage}[t]{\textwidth-\widthof{\underline{\textbf{Postulado \thepostula}}. }} #1 \itshape "}{"\end{minipage}\end{mybox}}  %%definicion de entorno de postulado

\newtheorem{theorem}{Teorema}
\usepackage{enumitem}

\begin{document}

\begin{center}
   \HUGE\textbf{ \scshape Mecánica Teórica}  \\ \huge Entregable del tema 2 \\ \vspace{0.5cm}\today \\
   \vspace{0.7cm}
   \large Manuel Lozano Bermúdez, María Palfy Alonso-Alegre, \\ Sergio Zucchi Mesia,  Alberto Mateo Martín.\\ \rule{\textwidth}{2pt} 
\end{center}

\begin{mybox}
        \begin{enumerate}
        \setcounter{enumi}{1}
        \item An NH3 molecule is described, in a simplified description, by two states $|U\rangle, |D\rangle$ which correspond to the position of the N atom to be above, or below, the plane formed by the H atoms. In this basis, the hamiltonian is defined by the matrix
        $$
        H=      \begin{pmatrix}
                    A & B\\
                    B & A
                \end{pmatrix}
        $$
            \begin{itemize}
                \item[(a)] Obtain the expression of the eigenstates and the eigenvalues of the hamiltonian, in terms of the constants A and B. Give the numerical values of the eigenvalues, with the proper units. Express the eigenstates $|I\rangle, |II\rangle$ in terms of the states $|U\rangle, |D\rangle$.
                \item[(b)] Give the expression of the evolution operator in the basis of the eigenstates of the hamiltonian.
                \item[(c)] If initially, for $t = 0$, the $NH_3$ molecule is in the state $|U\rangle$, express the state, at an arbitrary time t, in the basis $|I\rangle, |II\rangle$ , and in the basis $|U\rangle, |D\rangle$.
                \item[(d)] Calculate the probability of measuring the N atom above or below the plane formed by the H atoms, after a time $t = 5\cdot 10^{-9} $s.
            \end{itemize}
        Data: $\hbar = 6.58 \cdot 1o^{-6} eV $s; $A=1.12eV$; $B=0.230 \cdot10^{-6}$eV
        \end{enumerate}
\end{mybox}
\emph{\bfseries Solution:} \\

\begin{mybox}
        \begin{enumerate}
        \setcounter{enumi}{1}
        \item An NH3 molecule is described, in a simplified description, by two states $|U\rangle, |D\rangle$ which correspond to the position of the N atom to be above, or below, the plane formed by the H atoms. In this basis, the hamiltonian is defined by the matrix
        $$
        H=      \begin{pmatrix}
                    A & B\\
                    B & A
                \end{pmatrix}
        $$
            \begin{itemize}
                \item[(a)] Obtain the expression of the eigenstates and the eigenvalues of the hamiltonian, in terms of the constants A and B. Give the numerical values of the eigenvalues, with the proper units. Express the eigenstates $|I\rangle, |II\rangle$ in terms of the states $|U\rangle, |D\rangle$.
                \item[(b)] Give the expression of the evolution operator in the basis of the eigenstates of the hamiltonian.
                \item[(c)] If initially, for $t = 0$, the $NH_3$ molecule is in the state $|U\rangle$, express the state, at an arbitrary time t, in the basis $|I\rangle, |II\rangle$ , and in the basis $|U\rangle, |D\rangle$.
                \item[(d)] Calculate the probability of measuring the N atom above or below the plane formed by the H atoms, after a time $t = 5\cdot 10^{-9} $s.
            \end{itemize}
        Data: $\hbar = 6.58 \cdot 1o^{-6} eV $s; $A=1.12eV$; $B=0.230 \cdot10^{-6}$eV
        \end{enumerate}
\end{mybox}
\emph{\bfseries Solution:} \\

\begin{mybox}
        \begin{enumerate}
        \setcounter{enumi}{1}
        \item A Silver atom, which is a spin $1/2$ particle, with a mass $m_{Ag} = 100.479 GeV/c^2$ with a magnetic moment $\mu= 1.0\mu_B$ and its spin initially in the positive direction along the z axis, enters a homogeneous magnetic field, whose value is $B = 0.009 T$ along an axis in the x-z plane that forms $60$ deg with the z axis. The Hamiltonian that describes the interaction inside the magnet, in the basis of spin projections along the z axis $\{|+z\rangle,|-z\rangle\}$, is
        $$
        H=B\mu \begin{pmatrix}
                    1/2 & \sqrt{3/4}\\
                    \sqrt{3/4} & 1/2
                \end{pmatrix}
        $$
            \begin{itemize}
                \item[(a)] Draw a figure indicating the motion of the particle as well as the direction of the magnetic field. Describe qualitatively the motion of the particle: Does the particle move along a straight line, or does it deviate? Does the spin change?
                \item[(b)] Obtain the eigenstates and the eigenvalues of the hamiltonian. Give the numerical values of the eigenvalues, with the proper units. Express the eigenstates $|A\rangle,|B\rangle$ in terms of the states $|+z\rangle,|-z\rangle$.
                \item[(c)] Give the expression of the evolution operator in the basis of the eigenstates of the hamiltonian.
                \item[(d)] Express the initial state in terms of the eigenstates $|A\rangle,|B\rangle$.
                \item[(e)] Express the state, at an arbitrary time t, in terms of the eigenstates $|A\rangle,|B\rangle$.
                \item[(f)] Calculate the probability of measuring positive and negative spin projections along the z axis, after a time $t = 5\cdot 10^{-9} $s.
            \end{itemize}
        Data: $\hbar = 6.58 \cdot 1o^{-6} eV s; \mu_B = 5.788 \cdot10^{-5} eV/T$
        \end{enumerate}
\end{mybox}
\emph{\bfseries Solution:} \\


\begin{mybox}
        \begin{enumerate}
        \setcounter{enumi}{1}
        \item An NH3 molecule is described, in a simplified description, by two states $|U\rangle, |D\rangle$ which correspond to the position of the N atom to be above, or below, the plane formed by the H atoms. In this basis, the hamiltonian is defined by the matrix
        $$
        H=      \begin{pmatrix}
                    A & B\\
                    B & A
                \end{pmatrix}
        $$
            \begin{itemize}
                \item[(a)] Obtain the expression of the eigenstates and the eigenvalues of the hamiltonian, in terms of the constants A and B. Give the numerical values of the eigenvalues, with the proper units. Express the eigenstates $|I\rangle, |II\rangle$ in terms of the states $|U\rangle, |D\rangle$.
                \item[(b)] Give the expression of the evolution operator in the basis of the eigenstates of the hamiltonian.
                \item[(c)] If initially, for $t = 0$, the $NH_3$ molecule is in the state $|U\rangle$, express the state, at an arbitrary time t, in the basis $|I\rangle, |II\rangle$ , and in the basis $|U\rangle, |D\rangle$.
                \item[(d)] Calculate the probability of measuring the N atom above or below the plane formed by the H atoms, after a time $t = 5\cdot 10^{-9} $s.
            \end{itemize}
        Data: $\hbar = 6.58 \cdot 1o^{-6} eV $s; $A=1.12eV$; $B=0.230 \cdot10^{-6}$eV
        \end{enumerate}
\end{mybox}
\emph{\bfseries Solution:} \\

\end{document}

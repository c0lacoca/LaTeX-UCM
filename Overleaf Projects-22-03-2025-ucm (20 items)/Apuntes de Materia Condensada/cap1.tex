\chapter{Electrones con interacción}
\section{Introducción}
Comenzaremos estableciendo las aproximaciones básicas que usaremos a lo largo de la asignatura. En primer lugar, hablaremos de los electrones independientes. En esta aproximación, supondremos que los electrones no interactúan entre sí y únicamente se ven influenciados por la actuación de un potencial externo. Bajo esta premisa, los electrones independientes tienen que satisfacer el siguiente problema de autovalores:
\begin{align}\label{1.1}
-\frac{\hbar^2}{2m}\laplacian \psi_i (\B{r}) + U(\B{r}) \psi_i(\B{r}) = \varepsilon_i \psi_i(\B{r})\ ,\\
T = 0 \qquad \varepsilon_1\le \varepsilon_2 \le \ldots \le \varepsilon_N\ .
\end{align}
Cada nivel electrónico tendrá una energía $\varepsilon_i$, y los electrones irán ocupando los $N$ niveles de energía. A estos niveles se les denomina \emph{one-electron state}. El estado con mayor energía se denomina \textbf{nivel de Fermi}, y se denota como $\varepsilon_F$. Esta es una de las aproximaciones que hicimos en estado sólido, pero el verdadero problema que trataremos será el problema de varios cuerpos (\emph{many-body state}) en el cual cada electrón vendrá determinado por su posición y su espín
\begin{equation}
    (\B{r}_i,\sigma_i); \ \B{r}_i \in \K{R}^3;\ \sigma_i \in \left \{ \frac{1}{2},-\frac{1}{2} \right  \} \equiv \{ +,- \} = \{ \uparrow,\downarrow \} \ .
\end{equation}
Para este sistema de $N$ electrones, la función de onda global $\Psi(\{ \B{r}_i \sigma_i  \})$ debe cumplir la ecuación de Schrödinger independiente del tiempo,
\begin{equation}
    H \Psi = \sum_i ^N \left ( -\frac{\hbar^2 }{2m}\laplacian_i \Psi - \sum_{\B{R}}\frac{Ze^2}{|\B{r}_i - \B{R}|} \Psi \right ) + \frac{1}{2} \sum_{i\neq j} \frac{e^2}{|\B{r}_i - \B{r}_j|} \Psi = E_T \Psi\ .
\end{equation}
Notemos que el término con $Ze^2$ se debe a los electrones que aportan cada átomo (en nuestro caso, metales monoatómicos). El otro término de interacción se debe a las propias repulsiones coulombianas que sienten los electrones en $\B{r}_i$ y $\B{r}_j$. Como solamente tenemos en cuenta una pareja, tenemos que dividir por la mitad para quedarnos con las que nos interesan. La energía de este sistema many-body será $E_T$. \\

Asumimos que los iones que aportan los electrones permanecen estáticos en las posiciones $\B{R}\in B$ (espacio de red), pero sabemos que las fluctuaciones o vibraciones de estos iones dan lugar a los fonones o cuantos de vibración que estudiamos en estado sólido. Veremos esta situación más adelante. \\

El problema de $N$ cuerpos es intratable tanto analíticamente como de forma numérica, de modo que es necesario realizar nuevas aproximaciones para hacerlo resoluble al menos a nivel computacional. Una aproximación razonable puede ser la del campo medio: tomaremos que el potencial $U(\B{r})$ visto por un electrón estará constituido por dos contribuciones medias debidas al potencial producido por los iones, $U^{ion}$, y por los electrones, $U^{el}$. El potencial iónico será sencillamente la repulsión coulombiana vista anteriormente, y el potencial electrónico se deberá a la densidad de carga producida por los electrones, 
\begin{align}
    U(\B{r}) &= U^{ion} + U^{el} \ ,& U^{ion} &= -\sum_{\B{R}} \frac{Ze^2}{|\B{r} - \B{R}|}\ , \\
             &                   & U^{el}  &= -e\int \D[3]{\B{r'}} \frac{\rho(\B{r'})}{|\B{r} - \B{r'}|}\ ; 
\end{align}
donde tomaremos la densidad de carga electrónica como $\rho(\B{r}) = \sum_i \rho_i (\B{r}) = -e\sum_i |\psi_i (\B{r})|^2$. Notamos que dentro del potencial electrónico hemos incluido la interacción del propio electrón consigo mismo. Aunque esto, en principio, sea incorrecto, a todos los efectos prácticos será despreciable y se podrá mantener esta expresión. \\

Una vez realizada esta aproximación, podemos plantear finalmente la ecuación para la f.d.o del estado monoelectrónico como la ecuación de Schrödinger  con el potencial de campo medio planteado,
\begin{equation}
    -\frac{\hbar^2 }{2m}\laplacian\psi_i + U^{ion}(\B{r})\psi_i (\B{r}) + e^2 \sum_j \int \D{\B{r'}} \frac{|\psi_j(\B{r'})|^2}{|\B{r} - \B{r'}|} \psi_i (\B{r}) = \varepsilon_i \psi_i (\B{r})\ .
\end{equation}
En este punto se hace evidente que hay un problema: el argumento se ha vuelto circular, ya que para encontrar la función de onda del estado monoelectrónico, necesitamos conocer la densidad de carga, que a su vez requiere de conocer de primera mano la función de onda. La manera de obtener esta f.d.o es mediante un método iterativo auto-consistente. Partiremos de la suposición de una f.d.o inicial (que pueden ser, por ejemplo, los orbitales atómicos del átomo de hidrógeno), obtendremos la densidad de carga a partir de ellos y resolveremos la ecuación para obtener las nuevas f.d.o. Una vez hecho eso, repetiremos este proceso con las nuevas funciones, así hasta alcanzar la auto-consistencia (es decir, hasta que las f.d.o de partida sean iguales a las de salida). Este proceso iterativo se denomina \textbf{aproximación de Hartree}, y no tiene en cuenta los siguientes efectos
\begin{enumerate}
    \item Efectos de \textbf{canje} (exchange): La indistinguibilidad de los electrones juega un papel en las funciones de onda que obtenemos, ya que imponen condiciones sobre la simetría que deben de tener.
    \item Efectos de \textbf{apantallamiento} (screening): Los electrones apantallan las cargas de los iones, dando lugar a diferentes interacciones y, por lo tanto, diferentes potenciales.
    \item Efectos de \textbf{interacción electrón-electrón}, descritos por la teoría de Landau, que proporciona el entendimiento correcto de la aproximación de electrones independientes. 
\end{enumerate}
\subsection{Aproximación de Hartree y efectos de canje}
Las ecuaciones de Hartree se pueden obtener si minimizamos el valor esperado de la energía del sistema many-body.
\begin{flalign}
    \expval{H}_\Psi &= \frac{\matrixelement{\Psi}{H}{\Psi}}{\braket{\Psi}}\\
    \text{con } \braket{\Psi}{\Phi} &= (\Psi,\Phi) \equiv \sum_{\sigma_1,\ldots ,\sigma_N} \int \D{\B{r}_1}\cdots \D{\B{r}_N} \Psi^*(\B{r}_1\sigma _1,\ldots , \B{r}_N \sigma_N) \Phi(\B{r}_1 \sigma_1,\ldots )\ .
\end{flalign}
\paragraph{Notación} De aquí en adelante adoptaremos la siguiente notación: $\xi_i \equiv \B{r}_i \sigma_i$; de manera que $\Psi(\xi_1,\xi_2,\ldots) = \psi_1(\xi_1)\psi_2(\xi_2)\cdots\psi_N(\xi_N)$. El índice de la función de onda, $\psi_\lambda$, representa el estado del electrón i-ésimo con número orbital $k$ y número cuántico de espín $\alpha $: $\lambda = k,\alpha $.\\

Como hemos dicho anteriormente, la función del problema many-body debe tener en cuenta la indistinguibilidad del electrón, que en este caso se manifiesta a través del ppo. de exclusión de Pauli como la función de onda global antisimetrizada bajo intercambio de electrones por parejas,
\begin{equation}
    \Psi (\ldots , \xi_i , \xi_j,\ldots) = -\Psi(\ldots,\xi_j,\xi_i,\ldots)\ .
\end{equation}
Por ello, la función de onda global tiene que ser de la forma de \textbf{determinante de Slater},
\begin{equation}
    \Psi = \frac{1}{\sqrt{N!}} \left [\text{det. Slater} \right ] = \frac{1}{\sqrt{N!}} \left | \begin{array}{cccc}
    \psi_1(\xi_1)     &\psi_1(\xi_2)  &\cdots  &\psi_1(\xi_N)  \\
    \psi_2(\xi_1)     &\psi_2(\xi_2)  &\cdots  &\psi_2(\xi_N)  \\
     \vdots           &\vdots         &\ddots  &\vdots         \\
    \psi_N(\xi_1)     &\psi_N(\xi_2)  &\cdots  &\psi_N(\xi_N)  \\
    \end{array} \right | \ ,
\end{equation}
suponiendo que los estados monoelectrónicos son \textbf{ortonormales}: $\braket{\psi_i}{\psi_j} = \var_{ij} = \var_{k_ik_j} \var_{\alpha_i \alpha_j}$. Introduciendo esta $\Psi$ en el valor esperado de $H$:
\begin{align}
    \expval{H}_\Psi &= \textcolor{red}{\sum_i \sum_\sigma \int \D{\B{r}} \ \psi_i ^*(\B{r}\sigma) \left [  -\frac{\hbar^2 }{2m}\laplacian + U^{ion}(\B{r}) \right ] \psi_i (\B{r} \sigma )} \notag\\
                     & \textcolor{blue}{+\frac{1}{2} \sum_{i,j} \sum_{\sigma ,\sigma'} \int \D{\B{r},\B{r'}}\ \frac{e^2}{|\B{r} - \B{r'}|} |\psi_i (\B{r}\sigma )|^2 |\psi_j (\B{r'}\sigma ')|^2} \notag\\
                     & \textcolor{violet}{-\frac{1}{2} \sum_{i,j} \sum_{\sigma,\sigma'} \int \D{\B{r},\B{r'}} \ \frac{e^2}{|\B{r} - \B{r'}|} \delta_{\alpha_i\alpha_j} \psi_i ^*(\B{r}\sigma )\psi_i(\B{r'}\sigma') \psi^*_j(\B{r'}\sigma') \psi_j (\B{r}\sigma )} \ .
\end{align}
El término en \textcolor{red}{rojo} representa la contribución \emph{one-electron} que mencionamos previamente, el término en \textcolor{blue}{azul} el un término de interacción directa (Hartree) y el término en \textcolor{violet}{violeta} es un término de canje (Fock) entre los estados $i$ y $j$. Por las propiedades de la delta de kronecker, podemos reducir el término de canje,
\[  
-\frac{1}{2} \sum_{i,j} \sum_{\sigma,\sigma'} \int  \D{\B{r},\B{r'}} \ \frac{e^2}{|\B{r} - \B{r'}|}  \psi_i ^*(\B{r}\sigma )\psi_i(\B{r'}\sigma) \psi^*_j(\B{r'}\sigma) \psi_j (\B{r}\sigma )\qquad  \text{\footnotesize (\( \sigma' \to \sigma  \))}
\]
Para que en nuestra energía aparezca ese término de cambio del que hablamos, es preciso que exista una región del espacio en la cual las f.d.o one-electron de los estados $i$ y $j$ solapen. Es decir, \( \exists \ \Gamma : \B{r} \in \Gamma ,\quad \psi_i(\B{r}),\psi_j(\B{r}) \neq 0   \). Este canje se produce además entre electrones con misma componente de espín. 